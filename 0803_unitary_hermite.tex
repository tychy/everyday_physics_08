\documentclass[twocolumn]{jsarticle}
\usepackage{amssymb,amsmath,amsthm}
\usepackage{newtxtt}
\usepackage[utf8]{inputenc}
\newtheorem{th.}{Statement}
\newcommand{\pder}[2][]{\frac{\partial#1}{\partial#2}}
\newcommand{\dder}[2][]{\frac{\mathrm{d}#1}{\mathrm{d}#2}}
\newcommand{\ppder}[2][]{\frac{\partial^2#1}{{\partial#2}^2}}
\newcommand{\pikder}[3][]{\frac{\partial^2#1}{{\partial#2 \partial#3}}}
\newcommand{\pikdergx}[3][]{\frac{\partial^2 g_{#1}}{{\partial x^{#2} \partial x^{#3}}}}
\newcommand{\pderx}[2][]{\pder[#1]{x^{#2}}}
\newcommand{\pdergx}[2][]{\pderx[g_{#1}]{#2}}
\newcommand{\half}{\frac{1}{2}}
\newcommand{\hfpt}{\hspace{5pt}}
\newcommand{\ddfrac}[2]{\frac{{#1}^2}{{#2}^2}}
\newcommand{\beq}{\begin{equation}}
\newcommand{\beql}[1]{\begin{equation}\label{#1}}
\newcommand{\eeq}{\end{equation}}
\newcommand{\eeqp}{\;\;\;.\end{equation}}
\newcommand{\eeqc}{\;\;\;,\end{equation}}
\newcommand{\GaT}[3]{\Gamma^{#1}_{#2 #3}}
\newcommand{\pderGaTx}[4]{\pderx[\GaT{#1}{#2}{#3}]{#4}}
\newcommand{\Christfinside}[3]{\pdergx[#3 #1]{#2} + \pdergx[#2 #3]{#1} - \pdergx[#1 #2]{#3}}
\newcommand{\Christf}[4]{\Gamma^{#1}_{#2 #3}=\half g^{#1 #4}(\Christfinside{#2}{#3}{#4})}
\newcommand{\Ricchiinside}[2]{\pder[\Gamma^l_{#1 #2}]{x^l} - \pder[\Gamma^l_{#1 l}]{x^{#2}} 
    + \GaT{l}{#1}{#2}\GaT{m}{l}{m} - \GaT{m}{#1}{l}\GaT{l}{#2}{m}}
\date{\today}
\author{山田龍}
\title{ユニタリー行列をエルミート行列で表す}
\begin{document}
\maketitle
\section{}
ほとんどの議論がEMANによる
ユニタリー行列の定義
\beq
    U^\dagger U = 1
\eeq
$U^\dagger = U^{-1}$が同様に成り立つ。
ユニタリ行列は別にユニタリ行列で必ず対角化できる。同時に対角化された行列もユニタリ行列である。
\begin{align}
    U^{\prime \dagger} U &= (V^{-1}UV)^{\dagger}(V^{-1}UV)\\
                         &= (V^{\dagger}U^{\dagger}(V^{-1})^{\dagger})(V^{-1}UV)\\
            &= (V^{\dagger}U^{\dagger}V)(V^{-1}UV)\\
            &= 1
\end{align}
この変換されたユニタリー行列は対角行列だが、各成分のノルムが1に等しいことがすぐにわかる。
\beq
    U_ii = e^{i\theta_i}
\eeq
と定義できる。これをテイラー展開すれば
\beq
U_ii = 1 + i\theta_i + \half (i\theta_i)^2 + \cdots
\eeq
行列全体で見れば、
\beq
    U^\prime = e^{iH^\prime}
\eeq
\beq
H^\prime
= \left(
        \begin{array}{ccc}
                  \theta_1 & 0 & 0 \\
                        0 & \theta_2 & 0 \\
                              0 & 0 & \ddots
        \end{array}
    \right)
\eeq
これがエルミートになっていることを見る。$U^{\prime}$を$U$に戻す。
\begin{align}
    U &= V U^\prime V^{-1}\\
      &= V e^{iH^\prime}V^{-1}\\
    &= e^{iVH^\prime V^{-1}}\\
    &= e^{iH}
\end{align}
\begin{align}
    H^\dagger &= (VH^\prime V^{-1})^\dagger\\
              &= VH^\prime V^{-1}
              &= H
\end{align}
$H^\prime$の成分は実数。Vはユニタリーであることを使った。
\subsection{tracelessなユニタリー行列}
行列式が1ならばtracelessなユニタリー行列で有ることを見る。
\beq
det U = det (V U^\prime V^{-1}) = det(V)det(U^\prime)det(V^{-1}) = det (U^\prime)
\eeq
つまり対角化した行列の行列式も等しいことになる。
\beq
    det U^\prime =  e^{i\theta_1}e^{i\theta_2}\cdots = 1
\eeq
よって、
\beq
    \sum \theta_i = 0
\eeq
であり、
\beq
    tr(H^\prime) = 0
\eeq
ここで
\beq
    tr(AB) = tr(BA)
\eeq
を使うと、
\beq
    tr(H^\prime) = tr(V^{-1} H V) = tr(V^{-1} V H) = tr(H)
\eeq
から$H$もトレースレスであることがわかった。例えば二行に列ならこれはSU(2)に当たり、パウリ行列が出てくる。
\end{document}

